\documentclass[xcolor={dvipsnames,usenames}]{beamer}
%\documentclass[xcolor={dvipsnames,usenames},handout]{beamer} % use this to compile w/o pauses
\mode<presentation>
{
\usetheme{madrid}
\usecolortheme{default}
\setbeamertemplate{itemize items}[default]
\setbeamertemplate{enumerate items}[default]
\setbeamercovered{transparent}
\usefonttheme[onlymath]{serif}
}


\usepackage{microtype}
\usepackage[normalem]{ulem}
%\usepackage{hyperref}
%\hypersetup{colorlinks=true, urlcolor=Blue, citecolor=Green, linkcolor=BrickRed, breaklinks, unicode}

\usepackage[nocompress]{cite}
\usepackage{amsmath,mathtools,stmaryrd}
\usepackage{enumerate}
\usepackage{latexsym,amsmath}
\usepackage{amssymb,stmaryrd}
\usepackage{euscript}
\usepackage{algorithm,algorithmicx}
\usepackage[noend]{algpseudocode}
\renewcommand{\algorithmicrequire}{\textbf{Requirement:}}

\usepackage{mathtools} % for \coloneqq

\usepackage{graphicx}
\usepackage{epstopdf}
\usepackage{subcaption}
\graphicspath{{./fig/}}


\title{Efficient Algorithms for Geometric Partial Matching}
\author[Allen Xiao]
{
	Pankaj~K.~Agarwal \and
	Hsien-Chih~Chang \and
	Allen~Xiao
}
\institute[SoCG 2019]
{
	Duke University
}
\date{June 2019}

% tweak wd=X\paperwidth to modify the footer dimensions in the madrid theme
\makeatletter
\setbeamertemplate{footline}{
  \leavevmode%
  \hbox{%
  \begin{beamercolorbox}[wd=.25\paperwidth,ht=2.25ex,dp=1ex,center]{author in head/foot}%
    \usebeamerfont{author in head/foot}\insertshortauthor\expandafter\ifblank\expandafter{\beamer@shortinstitute}{}{~~(\insertshortinstitute)}
  \end{beamercolorbox}%
  \begin{beamercolorbox}[wd=.55\paperwidth,ht=2.25ex,dp=1ex,center]{title in head/foot}%
    \usebeamerfont{title in head/foot}\insertshorttitle
  \end{beamercolorbox}%
  \begin{beamercolorbox}[wd=.2\paperwidth,ht=2.25ex,dp=1ex,right]{date in head/foot}%
    \usebeamerfont{date in head/foot}\insertshortdate{}\hspace*{2em}
    \insertframenumber/\inserttotalframenumber \hspace*{2ex}
  \end{beamercolorbox}}%
  \vskip0pt%
}
\makeatother


% make tables less packed
\renewcommand{\arraystretch}{1.5}
% get rid of caption labels
%\setbeamertemplate{caption}{\raggedright\insertcaption\par}
%\captionsetup[subfigure]{labelformat=empty}
% set beamer highlight color (alert)
\setbeamercolor{alerted text}{fg=BrickRed}

\newcommand{\mycite}[1]{{\color{LimeGreen}\lbrack #1\rbrack}}
\newcommand{\etal}{\textit{et~al.}}
\newcommand{\softO}{\widetilde{O}}
\newcommand{\reals}{\mathbb{R}}
\newcommand{\ints}{\mathbb{N}}
\newcommand\nats{\mathbb{N}}
\newcommand{\eps}{\varepsilon}
\DeclareMathOperator{\polylog}{polylog}
\DeclareMathOperator{\poly}{poly}
\newcommand{\flr}[1]{{\lfloor #1\rfloor}}
\DeclareMathOperator*{\argmax}{arg\,max}
\DeclareMathOperator*{\argmin}{arg\,min}
\DeclareMathOperator{\Vor}{Vor}
\DeclareMathOperator{\VorRegion}{VorRegion}

\def\abs#1{\mathopen| #1 \mathclose|}		% use instead of $|x|$
\def\norm#1{\mathopen\| #1 \mathclose\|}	% use instead of $\|x\|$

\DeclareMathOperator{\cost}{cost}
\newcommand{\tsupply}{\lambda}
\newcommand{\fsupply}{\phi}

\newcommand{\A}{{\color{red}A}}
\newcommand{\B}{{\color{blue}B}}
\newcommand{\M}{\EuScript{M}}
\newcommand{\tildeM}{\widetilde{\EuScript{M}}}
\newcommand{\X}{\EuScript{X}}

\def\EMPH#1{\textcolor{BrickRed}{{\emph{#1}}}}



\begin{document}


\begin{frame}
\maketitle
\end{frame}


% outline

% 01: gentle example, emphasize cost fn
% 02: unit-cap MCF form
% 03*: select prior work (non-geom, non-partial), emphasize SA
% 04: new results for partial, geom. based on prior work. present 1,2 only
% 05: Hungarian algorithm and least-cost augmentation
% 06: Hungarian search with BCP
% 07: HS: number of relaxations --> O(knpolylogn) time
% 08*: problem: initializing the BCP data structure (why not persistence?)
% 09*: rewinding --> k^2polylogn time per augmentation, one-time npolylogn pre
% 10: potential changes + Vaidya?
% 11: cost-scaling unit-cap MCF --- choose a different set of augmenting paths
% 12: changing scales (2\eps to \eps, losing circulation)
% 13: refinement by blocking flows (more of the same, \sqrt{k} per scale, O(k) size per blocking)
% 14*: new problem: number of relaxations could be \Omega(n) (example); compare vs Hung
% 15: dead/alive nodes, alive paths. "really want to query minimizing alive path instead", "don't track their potential"
% 16: alive paths in the BCP
% 17: # relaxations = O(|supp(f)|) = O(k)
% 18*: dead node potentials
% 19: punchline running time: per-scale npolylogn pre, k\sqrt{k}polylogn per, O(log(n^q/\eps)) scales
% 20: 




%TODO







%SECTION: pmt

% gentle example + equation
\begin{frame}{Example}
% point sets A and B of size m, n respectively
% allowed to translate A --> A+t
% minimum-cost k-matching between A+t and B
% what's the best translation?
\begin{figure}
\begin{center}
\includegraphics<1>[width=\textwidth,page=1]{pmt_example}%
\includegraphics<2>[width=\textwidth,page=2]{pmt_example}%
\includegraphics<3>[width=\textwidth,page=3]{pmt_example}%
\includegraphics<4>[width=\textwidth,page=4]{pmt_example}%
\includegraphics<5>[width=\textwidth,page=5]{pmt_example}%
\includegraphics<6>[width=\textwidth,page=6]{pmt_example}%
\includegraphics<7->[width=\textwidth,page=7]{pmt_example}%
\end{center}
\end{figure}
\begin{itemize}
\item<7-> Find the minimum-cost matching over all translations.
% in words, mention polytime algorithms when t fixed ``well studied problem''
\end{itemize}
\end{frame}





% end slide
\begin{frame}{The End}
\begin{center}
	Thank you.
\end{center}
\end{frame}

%\begin{frame}[allowframebreaks]{Citations}
%\tiny
%\bibliography{ref}
%\bibliographystyle{alpha}
%\end{frame}


\end{document}
